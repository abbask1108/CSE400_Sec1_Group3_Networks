\documentclass[12pt]{article}
\usepackage[utf8]{inputenc}
\usepackage{amsmath, amssymb, amsfonts}
\usepackage{geometry}
\usepackage{enumerate}

\geometry{a4paper, margin=1in}

\title{CSE 400: Fundamentals of Probability in Computing \\ \large Tutorial-1: Important Formulae and Theories}
\author{}
\date{}

\begin{document}

\maketitle

\section*{Q1. Division into Unlabeled Groups (Multinomial Counting)}

\subsection*{(a) Total number of ways}
\textbf{Given:}
\begin{itemize}
    \item 20 distinct dishes
    \item Divided into 4 groups
    \item Each group contains 5 dishes
    \item Order within groups and order of groups does not matter
\end{itemize}

Using the multinomial formula adjusted for unlabeled groups:
\[ \text{Total ways} = \frac{20!}{(5!)^4 4!} \]

\subsection*{(b) Probability all platters contain same cuisine}
\textbf{Since:}
\begin{itemize}
    \item Each cuisine has exactly 5 dishes.
    \item Only one specific grouping (by cuisine) satisfies the condition.
    \item Platters are unlabeled.
\end{itemize}
\[ P = \frac{1}{\frac{20!}{(5!)^4 4!}} = \frac{(5!)^4 4!}{20!} \]

\hr

\section*{Q2. Uniform Distribution (Waiting Time)}
Arrival is uniformly distributed over 30 minutes.
\[ P = \frac{\text{Favorable time}}{30} \]

\textbf{(a) Waiting < 5 minutes} \\
Total favorable time = 10 minutes.
\[ P(\text{wait} < 5) = \frac{10}{30} = \frac{1}{3} \]

\textbf{(b) Waiting > 10 minutes} \\
Total favorable time = 10 minutes.
\[ P(\text{wait} > 10) = \frac{10}{30} = \frac{1}{3} \]

\hr

\section*{Q3. Conditional Probability \& De Morgan’s Law}
Let $L$ be the event "arrives late" and $E$ be the event "leaves early." \\
\textbf{Given:} $P(L)=0.15, P(E)=0.25, P(L \cap E)=0.08$.

\begin{itemize}
    \item $P(E^c) = 1 - P(E) = 0.75$
    \item $P(L \cup E) = P(L) + P(E) - P(L \cap E) = 0.32$
    \item \textbf{De Morgan’s Law:} $P(L^c \cap E^c) = 1 - P(L \cup E) = 0.68$
\end{itemize}

\textbf{Conditional probability:}
\[ P(L^c | E^c) = \frac{P(L^c \cap E^c)}{P(E^c)} = \frac{0.68}{0.75} \approx 0.907 \]

\hr

\section*{Q4. Poisson Distribution ($\lambda = 0.2$)}
Poisson PMF: $P(X=k) = \frac{e^{-\lambda}\lambda^k}{k!}$

\textbf{(a)} $P(X=0) = e^{-0.2} \approx 0.8187$ \\
\textbf{(b)} $P(X \ge 2) = 1 - P(0) - P(1)$
\[ P(1) = 0.2e^{-0.2} \approx 0.1637 \]
\[ P(X \ge 2) = 1 - 0.8187 - 0.1637 = 0.0176 \approx 0.0175 \]

\hr

\section*{Q5. Poisson Distribution ($\lambda = 3.5$)}
\textbf{(a)} $P(X \ge 2) = 1 - P(0) - P(1)$
\begin{itemize}
    \item $P(0) = 0.0302$
    \item $P(1) = 0.1057$
\end{itemize}
\[ P(X \ge 2) = 0.8641 \]
\textbf{(b)} $P(X \le 1) = P(0) + P(1) = 0.1359$

\hr

\section*{Q6. Poisson Distribution ($\lambda = 3$)}
\textbf{(a)} $P(X \ge 3) = 1 - [P(0) + P(1) + P(2)]$
\begin{itemize}
    \item $P(0) = 0.0498, P(1) = 0.1494, P(2) = 0.2240$
\end{itemize}
\[ P(X \ge 3) = 0.5768 \]
\textbf{(b) Conditional Probability:}
\[ P(X \ge 3 | X \ge 1) = \frac{P(X \ge 3)}{P(X \ge 1)} = \frac{0.5768}{1 - 0.0498} \approx 0.607 \]

\hr

\section*{Q7. Gaussian Distribution \& Q-Function}
Given PDF: $f_X(x) = \frac{1}{\sqrt{8\pi}} \exp\left(-\frac{(x+3)^2}{8}\right)$ \\
Comparing with $\frac{1}{\sqrt{2\pi\sigma^2}}\exp\left(-\frac{(x-m)^2}{2\sigma^2}\right)$, we identify:
\[ m = -3, \quad \sigma = 2 \]

\textbf{Results:}
\begin{itemize}
    \item $P(X \le 0) = \Phi\left(\frac{0 - (-3)}{2}\right) = 1 - Q(1.5)$
    \item $P(X > 4) = Q\left(\frac{4 - (-3)}{2}\right) = Q(3.5)$
    \item $P(|X+3| < 2) = 1 - 2Q(1)$
    \item $P(|X-2| > 1) = 1 - Q(2) + Q(3)$
\end{itemize}

\hr

\section*{Q8. Binomial Model (Jury Problem)}
Let $\theta$ be the probability an individual juror makes a correct decision. \\
If $\alpha = P(\text{guilty})$:
\[ \text{Total Correct} = \alpha \sum_{i=8}^{12} \binom{12}{i} \theta^i (1-\theta)^{12-i} + (1-\alpha)\sum_{i=5}^{12} \binom{12}{i} \theta^i (1-\theta)^{12-i} \]

\hr

\section*{Q9. Poisson PMF from Normalization}
Given $p(i) = \frac{c\lambda^i}{i!}$. Since $\sum_{i=0}^\infty p(i) = 1$ and $e^\lambda = \sum \frac{\lambda^i}{i!}$:
\[ ce^\lambda = 1 \implies c = e^{-\lambda} \]
\[ P(X > 2) = 1 - e^{-\lambda} - \lambda e^{-\lambda} - \frac{\lambda^2 e^{-\lambda}}{2} \]

\textbf{CDF of Discrete Random Variable:}
$F(a) = \sum_{x \le a} p(x)$. This is a step function with jumps at $x_i$ equal to $p(x_i)$.

\hr

\section*{Q10. Binomial Reliability Model}
\textbf{(a) 5-component vs 3-component:}
A majority must function.
\[ P_5 = 10p^3(1-p)^2 + 5p^4(1-p) + p^5 \]
\[ P_3 = 3p^2(1-p) + p^3 \]
The 5-component system is better if $P_5 - P_3 > 0$, which reduces to:
\[ 3(p-1)^2(2p-1) > 0 \implies p > \frac{1}{2} \]

\textbf{(b) General Case:}
\[ P_{2k+1} - P_{2k-1} = \binom{2k-1}{k} p^k(1-p)^k(2p-1) \]
Better performance $\iff p > \frac{1}{2}$.

\end{document}