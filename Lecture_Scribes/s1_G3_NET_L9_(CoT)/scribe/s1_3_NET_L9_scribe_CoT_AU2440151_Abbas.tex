\documentclass[12pt]{article}
\usepackage{amsmath, amssymb, graphicx}
\usepackage{geometry}
\geometry{margin=1in}

\title{Lecture 9 Scribe --- Continuous Random Variables}
\author{Course: CSE400 Fundamentals of Probability in Computing}
\date{Topic: Uniform, Exponential, Laplace and Gamma Random Variables \\ Source: Lecture slides (Feb 2, 2026)}

\begin{document}

\maketitle

\section{Lecture Overview and Scope}

The lecture introduces types of continuous random variables and focuses on:
\begin{itemize}
    \item Uniform random variable
    \item Exponential random variable
    \item Laplace random variable (listed in outline)
    \item Gamma random variable (listed with graph, special cases, example, homework)
    \item Problem solving and in-class activity
\end{itemize}

\section{Continuous Random Variables}

A continuous random variable (RV) is characterized through:
\begin{itemize}
    \item Probability Density Function (PDF)
    \item Cumulative Distribution Function (CDF)
    \item Applications and problem solving (mentioned in lecture annotations)
\end{itemize}

\section{Uniform Random Variable}

\subsection{Definition and Notation}

Let $X \sim \text{Uniform}(a,b)$.

\textbf{Probability Density Function (PDF)}
\[
f_X(x) =
\begin{cases}
\frac{1}{b-a}, & a \le x < b \\
0, & \text{elsewhere}
\end{cases}
\]

\textbf{Cumulative Distribution Function (CDF)}
\[
F_X(x) =
\begin{cases}
0, & x < a \\
\frac{x-a}{b-a}, & a \le x < b \\
1, & x \ge b
\end{cases}
\]

\subsection{Graphical Interpretation}

PDF: constant over $[a,b]$.

CDF: increases linearly from 0 to 1 over $[a,b]$.

\subsection{Example \#1 --- Uniform RV (Phase of Sinusoid)}

Problem

Phase uniformly distributed over $\theta \in [0,2\pi]$.

\[
f_{\theta}(\theta) =
\begin{cases}
\frac{1}{2\pi}, & 0 \le \theta < 2\pi \\
0, & \text{otherwise}
\end{cases}
\]

Tasks:
\begin{enumerate}
    \item $Pr(\theta > 3\pi/4)$
    \item $Pr(\theta < \pi \mid \theta > 3\pi/4)$
    \item $Pr(\cos\theta < 1/2)$
\end{enumerate}

\subsection{Solution --- Step-by-Step}

Key Property of Uniform Distribution

For uniform RV on $[a,b]$:
\[
Pr(a < \theta < b) = \frac{b-a}{2\pi}
\]

\textbf{(a) Compute $Pr(\theta > 3\pi/4)$}

Interval length:
\[
2\pi - \frac{3\pi}{4} = \frac{5\pi}{4}
\]

\[
Pr(\theta > 3\pi/4) = \frac{5\pi/4}{2\pi} = \frac{5}{8}
\]

\textbf{(b) Conditional Probability}

\[
Pr(A|B) = \frac{Pr(A \cap B)}{Pr(B)}
\]

Where:
\[
A: \theta < \pi, \quad B: \theta > 3\pi/4
\]

Intersection:
\[
3\pi/4 < \theta < \pi
\]

Length:
\[
\pi - \frac{3\pi}{4} = \frac{\pi}{4}
\]

\[
Pr(A \cap B) = \frac{\pi/4}{2\pi} = \frac{1}{8}
\]

\[
Pr(B) = \frac{5}{8}
\]

\[
Pr(\theta < \pi \mid \theta > 3\pi/4) = \frac{1/8}{5/8} = \frac{1}{5}
\]

\textbf{(c) Compute $Pr(\cos\theta < 1/2)$}

Solve:
\[
\cos\theta = \frac{1}{2}
\]

\[
\theta = \frac{\pi}{3}, \frac{5\pi}{3}
\]

Condition:
\[
\frac{\pi}{3} < \theta < \frac{5\pi}{3}
\]

Interval length:
\[
\frac{5\pi}{3} - \frac{\pi}{3} = \frac{4\pi}{3}
\]

\[
Pr(\cos\theta < 1/2) = \frac{4\pi/3}{2\pi} = \frac{2}{3}
\]

\subsection{Applications of Uniform Random Variable}

\begin{itemize}
    \item Phase of sinusoidal signals (angles equally likely)
    \item Random number generation in simulations
    \item Arrival time in a known window with no preference
\end{itemize}

\section{Exponential Random Variable}

\subsection{Definition}

For parameter $b>0$:

\textbf{PDF}
\[
f_X(x) = \frac{1}{b} e^{-x/b} u(x)
\]

\textbf{CDF}
\[
F_X(x) = [1 - e^{-x/b}] u(x)
\]

where $u(x)$ is the unit step function.

\subsection{Graphical Interpretation}

PDF: decays exponentially from maximum at $x=0$.

CDF: monotonically increases toward 1.

\subsection{Example \#2 --- Exponential RV}

Problem

Tasks:
\begin{enumerate}
    \item Find $Pr(3X < 5)$
    \item Generalize to $Pr(3X < y)$
\end{enumerate}

Step-by-Step Derivation

\textbf{(a) Compute $Pr(3X < 5)$}

Transform inequality:
\[
3X < 5 \Rightarrow X < \frac{5}{3}
\]

Using CDF:
\[
Pr(X < a) = F_X(a) = 1 - e^{-a}
\]

So:
\[
Pr(3X < 5) = 1 - e^{-5/3}
\]

\textbf{(b) Generalization}

(valid for $y>0$)
\[
Pr(3X < y) = Pr\left(X < \frac{y}{3}\right) = 1 - e^{-y/3}
\]

\section{Laplace and Gamma Random Variables}

Mentioned explicitly in lecture outline:

\begin{itemize}
    \item Laplace RV --- example to be discussed
    \item Gamma RV:
    \begin{itemize}
        \item graph and special cases
        \item example
        \item homework problem
    \end{itemize}
\end{itemize}

These are part of lecture scope though detailed derivations appear later or in extended material.

\section{Problem Solving Emphasis}

The lecture structure stresses:
\begin{itemize}
    \item working with PDFs and CDFs
    \item computing probabilities via interval lengths (uniform)
    \item using CDF transformation (exponential)
    \item applying conditional probability formulas
\end{itemize}

\section{Logical Flow of Lecture}

\begin{enumerate}
    \item Motivation: continuous distributions
    \item Uniform RV: definition $\rightarrow$ graphs $\rightarrow$ example $\rightarrow$ applications
    \item Exponential RV: definition $\rightarrow$ graphs $\rightarrow$ example
    \item Laplace \& Gamma introduced
    \item Problem solving and estimation tasks
\end{enumerate}

\section{Key Exam Takeaways}

\textbf{Uniform RV}
\begin{itemize}
    \item Constant PDF on interval
    \item Linear CDF
    \item Probability = interval length / total length
\end{itemize}

\textbf{Exponential RV}
\begin{itemize}
    \item PDF: decaying exponential
    \item CDF: $1 - e^{-x/b}$
    \item Probabilities via CDF substitution
\end{itemize}

\textbf{Conditional Probability Transformation}
\[
Pr(A|B) = \frac{Pr(A \cap B)}{Pr(B)}
\]

Convert inequalities before applying CDF:
\[
Pr(aX < b) \Rightarrow Pr\left(X < \frac{b}{a}\right)
\]

\end{document}
