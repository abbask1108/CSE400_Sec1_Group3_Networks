\documentclass{article}
\usepackage[utf8]{inputenc}
\usepackage{amsmath}
\usepackage{amssymb}
\usepackage{geometry}

\geometry{a4paper, margin=1in}

\begin{document}

\section*{CSE400 — Fundamentals of Probability in Computing}
\subsection*{Lecture 9: Uniform, Exponential, Laplace and Gamma Random Variables}
\textit{(Prepared as an exam-oriented lecture scribe from the provided lecture slides)}

\subsection*{1. Lecture Outline}
According to the outline slides, the lecture covers:
\begin{itemize}
    \item Types of Continuous Random Variables
    \item Uniform Random Variable — Example
    \item Exponential Random Variable — Example
    \item Laplace Random Variable — Example
    \item Gamma Random Variable
    \begin{itemize}
        \item Graph and Special Cases
        \item Example
        \item Homework Problem
    \end{itemize}
    \item Problem Solving
    \item In-class Activity: Gaussian Density Estimation
\end{itemize}
Only the material appearing in the provided slides is included in this scribe. \\
L9\_S1\_A

\subsection*{2. Types of Continuous Random Variables}
\subsubsection*{2.1 Uniform Random Variable}
\textbf{Definition (PDF)} \\
A continuous random variable $X$ is uniformly distributed over the interval $[a,b]$ if its probability density function (PDF) is given by
\[
f_X(x) = 
\begin{cases} 
\frac{1}{b-a}, & a \leq x < b, \\
0, & \text{elsewhere.}
\end{cases}
\]
This definition specifies:
\begin{itemize}
    \item The density is constant over the interval $[a,b]$.
    \item The density is zero outside the interval.
\end{itemize}

\textbf{Definition (CDF)} \\
The cumulative distribution function (CDF) is given by
\[
F_X(x) = 
\begin{cases} 
0, & x < a, \\
\frac{x-a}{b-a}, & a \leq x < b, \\
1, & x \geq b.
\end{cases}
\]
The slides show the corresponding PDF as a rectangle over $[a,b]$ and the CDF as a linear increase between $a$ and $b$ (Figure 3.8). \\
L9\_S1\_A

\subsection*{3. Example \#1 — Uniform Random Variable}
\textbf{Problem Statement} \\
The phase of a sinusoid, $\theta$, is uniformly distributed over $[0, 2\pi)$. The PDF is
\[
f_{\theta}(\theta) = 
\begin{cases} 
\frac{1}{2\pi}, & 0 \leq \theta < 2\pi, \\
0, & \text{otherwise.}
\end{cases}
\]
Tasks: \\
(a) Find $\Pr(\theta > 3\pi/4)$ \\
(b) Find $\Pr(\theta < \pi \mid \theta > 3\pi/4)$ \\
(c) Find $\Pr(\cos \theta < 1/2)$

\textbf{Given Property for Uniform RV} \\
For a uniform random variable on $[a,b]$,
\[
\Pr(a < \theta < b) = \frac{b-a}{2\pi}.
\]
This relation is used directly in the solution.

\textbf{(a) Compute $\Pr(\theta > 3\pi/4)$} \\
Step-by-step as shown in the slides:
\begin{itemize}
    \item The interval of interest is from $3\pi/4$ to $2\pi$.
    \item Length of interval: $2\pi - \frac{3\pi}{4}$
    \item Probability equals interval length divided by total length:
    \[ \Pr(\theta > 3\pi/4) = \frac{2\pi - 3\pi/4}{2\pi} = \frac{5}{8}. \]
\end{itemize}

\textbf{(b) Compute $\Pr(\theta < \pi \mid \theta > 3\pi/4)$} \\
The slides apply the conditional probability formula:
\[ \Pr(A \mid B) = \frac{\Pr(A \cap B)}{\Pr(B)}. \]
Steps:
\begin{itemize}
    \item Event $B$: $\theta > 3\pi/4 \implies \Pr(B) = \frac{5}{8}$.
    \item Event $A \cap B$: $3\pi/4 < \theta < \pi \implies \Pr(3\pi/4 < \theta < \pi) = \frac{\pi - 3\pi/4}{2\pi} = \frac{1}{8}$.
    \item Therefore, $\Pr(\theta < \pi \mid \theta > 3\pi/4) = \frac{1/8}{5/8} = \frac{1}{5}$.
\end{itemize}

\textbf{(c) Compute $\Pr(\cos \theta < 1/2)$} \\
Steps shown in the slides:
\begin{itemize}
    \item Solve $\cos \theta = \frac{1}{2}$. This occurs at $\theta = \frac{\pi}{3}, \frac{5\pi}{3}$.
    \item From the diagram in the solution, $\cos \theta < 1/2$ for $\frac{\pi}{3} < \theta < \frac{5\pi}{3}$.
    \item Probability: $\Pr(\cos \theta < 1/2) = \frac{5\pi/3 - \pi/3}{2\pi} = \frac{4\pi/3}{2\pi} = \frac{2}{3}$.
\end{itemize}

\textbf{Applications of Uniform Random Variable} \\
As listed in the slides:
\begin{itemize}
    \item Phase of a sinusoidal signal when all phase angles between $0$ and $2\pi$ are equally likely.
    \item Random number generated by a computer between 0 and 1 for simulations.
    \item Arrival time of a user within a known time window assuming no time preference.
\end{itemize}
L9\_S1\_A

\subsection*{4. Exponential Random Variable}
\textbf{Definition} \\
The exponential random variable has the following PDF and CDF (for any $b > 0$):

\textbf{PDF} \\
$f_X(x) = \frac{1}{b} \exp(-\frac{x}{b}) u(x)$

\textbf{CDF} \\
$F_X(x) = [1 - \exp(-\frac{x}{b})] u(x)$ \\
where $u(x)$ is the unit step function.

The slides also include plots showing:
\begin{itemize}
    \item A decreasing exponential PDF.
    \item A monotonically increasing CDF approaching 1.
\end{itemize}
L9\_S1\_A

\subsection*{5. Example \#2 — Exponential Random Variable}
\textbf{Problem Statement} \\
Let $X$ be an exponential random variable with PDF
\[ f_X(x) = e^{-x} u(x). \]
Tasks: \\
(a) Find $\Pr(3X < 5)$. \\
(b) Generalize the answer to find $\Pr(3X < y)$ for an arbitrary constant $y$. \\
(Only the problem statements are shown in the slides; solution steps are not provided in the visible material.)

\end{document}